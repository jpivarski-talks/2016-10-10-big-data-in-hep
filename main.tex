\documentclass[a4paper]{jpconf}
\usepackage{graphicx}
\bibliographystyle{iopart-num}
\begin{document}
\title{“Big Data” in HEP: A comprehensive use case study}

\author{Oliver Gutsche$^1$, Matteo Cremonesi$^1$, Peter Elmer$^2$, Bo Jayatilaka$^1$, Jim Kowalkowski$^1$, Jim Pivarski$^2$, Saba Sehrish$^1$, Cristina Mantilla Suárez$^1$, Alexey Svyatkovskiy$^2$, Nhan Tran$^1$}

\address{$^1$Fermi National Accelerator Laboratory, Batavia, IL, USA}
\address{$^2$Princeton University, Princeton, NJ, USA}

\ead{gutsche@fnal.gov}

\begin{abstract}
Experimental Particle Physics has been at the forefront of analyzing the world’s largest datasets for decades. The HEP community was the first to develop suitable software and computing tools for this task. In recent times, new toolkits and systems collectively called “Big Data” technologies have emerged to support the analysis of Petabyte and Exabyte datasets in industry. While the principles of data analysis in HEP have not changed (filtering and transforming experiment-specific data formats), these new technologies use different approaches and promise a fresh look at analysis of very large datasets and could potentially reduce the time-to-physics with increased interactivity.

In this talk, we present an active LHC Run 2 analysis, searching for dark matter with the CMS detector, as a testbed for “Big Data” technologies. We directly compare the traditional NTuple-based analysis with an equivalent analysis using Apache Spark on the Hadoop ecosystem and beyond. In both cases, we start the analysis with the official experiment data formats and produce publication physics plots. We will discuss advantages and disadvantages of each approach and give an outlook on further studies needed.
\end{abstract}

\section{Introduction}

In 2012, Particle Physics entered a new age. With the discovery of the Higgs Boson, the Standard Model was extended with the missing mechanism that gives rise to particle masses. This theory, developed since the 1960's and constrained by numerous experiments before discovery, followed a predictable path. Now that the Higgs Boson has been discovered, the way forward is wide open. Many different theories that could explain the shortcomings of the Standard Model need to be investigated.

Particle physics has always been at the forefront of analyzing the world's largest datasets. Although we constrain ourselves in this paper to High Energy Physics (HEP), where known particles are made to collide at the highest energies possible, the underlying data organization holds for all subfields of particle physics. The most basic concept of how data in HEP is organized is an event: all detector signals associated with a single beam crossing and high-energy collision. Events are the atomic unit of HEP data and may be processed separately, which is why the computational problems of particle physics can be easily parallelized. 

Events must be reconstructed to convert detector signals into measurements of particles produced in collisions. This is usually done centrally by each experiment. The reconstructed events are then used by different physics groups studying different questions. This process is very idiosyncratic, as individual researchers or groups are searching for different physics phenomenona in the data. This results in a challenging computational problem, where as many as thousands of physicists analyze the same datasets to extract different physics results.

The analysis process uses properties of the event such as energy or momentum of particles produced in the collisions. It is based on comparing the distribution of properties measured in many events with theoretical predictions of the same property, either calculated empirically or simulated using Monte Carlo techniques. Particle physics is a statistical science. Statistically significant numbers of recorded and simulated events are required to make claims.

The particle physics community was the first to develop suitable software and computing tools for this task. Since the vast majority of collisions result in uninteresting events and the rate is enormous, this community also developed procedures to avoid recording all particle collisions, collectively called the trigger. In this first pass, quantities of interest are calculated quickly to decide whether to save or discard the event in real time.

Trigger selections are as coarse as possible, so that analysis selections are likely to be their subsets. Each physics group analyzes their data by filtering unnecessary events (which we call "skimming") and eliminating unnecessary variables from the communal dataset (which we call "slimming") to get a manageable subsample. This subsample might be further filtered and transformed to optimize the statistical significance of the measurement or possible discovery. Understandably, this is an iterative process, which must be repeated in response to discoveries and mistakes.

With ever increasing data volumes in particle physics, this process gets harder to perform interactively, simply due to the time required to read, filter, and transform the data. In the future, exponentially growing datasets will make this problem acute. New techniques will be required if we are to continue exploring the nature of matter and the universe.

Recently, new toolkits and systems have emerged outside of the HEP community to analyze Petabyte and Exabyte datasets in industry, collectively called "Big Data." These new technologies use different approaches and promise a fresh look at analysis of extremely large datasets. In this paper, we present an active LHC Run 2 analysis, searching for dark matter with the CMS detector, as a testbed for the application of “Big Data” technologies to a HEP analysis problem. We directly compare the traditional analysis with an equivalent analysis performed in Apache Spark. In both cases, we start the analysis with the communal dataset and produce publication physics plots using traditional and Big Data technologies. 

\section{The Physics Use Case}

Today, we know that the universe consists primarily of matter that we do not understand. Only 4--5\% of the universe is made of matter described by the Standard Model$^{\lbrack\mbox{\scriptsize citation needed}\rbrack}$. The Standard Model describes 12 elementary particles: 3 leptons (electron, muon, tau), 3 neutrinos (electron-neutrino, muon-neutrino, tau-neutrino), and 6 quarks (up, down, strange, charm, bottom, top). It describes 4 force carriers or bosons: photon, W, Z, and gluons, as well as one Higgs Boson to provide mass to the leptons, quarks, and W/Z bosons. The remainder of the universe consists of (a) something invisible to all of these forces except perhaps the W and Z bosons, which is unlike any known particle and is called Dark Matter (25\%), and (b) something that appears to have negative pressure, called Dark Energy (70\%).

If Dark Matter {\it does} interact with the W or Z bosons (the Weak Force), then it would be produced in particle collisions, along with the visible particles well described by the Standard Model. Dark Matter particle(s) would propagate through the detector without being detected. However, the total momentum of the collision, transverse to the beam, would be preserved, leading to an apparent momentum imbalance among the visible particles, known as missing transverse energy.

The Dark Matter search conducted in this study is performed with the Large Hadron Collider (LHC)~\cite{lhc} at CERN~\cite{cern}. The LHC is the highest energy particle collider in the world, located in Geneva, Switzerland. It accelerates protons to an energy of 6.5 TeV in two circular, evacuated beam pipes. The beams of protons are brought to collision in four points around the almost 17 mile circumference ring. The Compact Muon Solenoid (CMS) Collaboration built, maintains, and operates the CMS detector at one of these collision points. The CMS detector consists of detector components that measure different properties of the particles produced in a collision, such as tracks left by charged particles and energy deposits from all particles that interact via photons and gluons. The detector components are arranged concentrically around the collision, like an onion, to provide the highest detail near the collision point. The dense distribution of subdetectors compared to other experiments of its type ("compact"), and the detailed instrumentation for measuring muons gives the 14,500 ton detector and the collaboration its name.

In particular, this Dark Matter search is looking for a signature in the events commonly referred to as "mono-X" where "X" can be a light quark or gluon, a vector boson, or a heavy quark such as a bottom or top quark. We focus our search on the “monoTop” signature, where the detectable particle is a single, unbalanced top quark.

\section{The Traditional Analysis Workflow}

The traditional user analysis workflow for CMS data applies two C++ frameworks to the centrally produced dataset: CMSSW, specially designed for analyzing CMS data, and ROOT, which is a general, experiment-independent C++ toolkit. The ROOT framework provides statistical tools and a serialization format to persist reconstructed and transformed objects in files.

Although these C++ frameworks can be very efficient, their organization can be difficult for end-users to understand. Moreover, it operates at a low level of abstraction, and even if a user knows how to process data efficiently, the difficulty in setting up a manual procedure may be outweighed by the time required to do so.

Most data analysts or analysis groups start by translating the class structure of the data into a "flat ntuple," in which events are rows of a table with primitive numbers or arrays of numbers as columns. These ntuples are written to files using the ROOT framework. They may then be analyzed without the CMSSW framework, with minimal dependencies.

Often, the ntuples are still too big for interactive analysis (seconds or minutes between operations, so that a data analyst can respond to results without losing his or her train of thought). Most analysis groups therefore introduce additional steps in which the ntuples themselves are skimmed and slimmed.

In the last step of the analysis, quantities from the final ntuple are aggregated and plotted as histograms. In some cases, data from qualitatively different datasets must be combined to make a single plot, for instance by stacking Monte Carlo contributions to the data (to see if the real data exceeds the sum of known processes) or combining Monte Carlo samples with different weights.

The time scale of the complete Dark Matter workflow, shown in Fig.~\ref{tadaaa}, can range from days to weeks, depending on how many reconstructed and simulated events are needed for the analysis. The first step is repeated about four times a year and the produced ntuples can be used by more than one analysis. The skimming and slimming step is executed between once and four times a month. The actual analysis step, producing plots, is repeated many times a day, since it represents the iterative and interactive analysis.

From a communal dataset of 200~Terabytes, the analysis ntuples have a size of about 2~Terabytes. These are then skimmed and slimmed down to several Gigabytes that are used for exploration and producing plots.

\section{The Spark Analysis Workflow on Commodity Hardware}

\section{The Spark Analysis Workflow on NERSC}

\section{The Comparison}

\section{The Lessons Learned}

\section{The Next Steps}
Start of next steps \cite{ex1}.

\section*{References}
\bibliography{main}

\end{document}